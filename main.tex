\documentclass[12pt,twocolumn]{article}
\usepackage{savetrees}
\usepackage{amsmath}
\usepackage{amsfonts}
\begin{document}
\title{Notes on How To Prove It: A Structured Approach}
\author{Brandon Koepke}
\maketitle
\section{Sentenial Logic}
\newcommand{\definition}[1]{\textbf{#1}}
\subsection{Deductive Reasoning and Logical Connectives}
\subsection{Truth Tables}
\definition{DeMorgan's laws}
\[
\lnot (P \land Q) \equiv \lnot P \lor \lnot Q
\]
\[
\lnot (P \lor Q) \equiv \lnot Q \land \lnot Q
\]
\definition{Commutative laws}
\[
P \land Q \equiv Q \land P
\]
\[
P \lor Q \equiv Q \lor P
\]
\definition{Associative laws}
\[
P \land (Q \land R) \equiv (P \land Q) \land R
\]
\[
P \lor (Q \lor R) \equiv (P \lor Q) \lor R
\]
\definition{Idempotent laws}
\[
P \land \equiv P
\]
\[
P \lor P \equiv P
\]
\definition{Distributive laws}
\[
P \land (Q \lor R) \equiv (P \land Q) \lor (P \land R)
\]
\[
P \lor (Q \land R) \equiv (P \lor Q) \land (P \lor R)
\]
\definition{Absorption laws}
\[
P \lor (P \land Q) \equiv P
\]
\[
P \land (P \lor Q) \equiv P
\]
\definition{Double Negation law}
\[
\lnot \lnot P \equiv P
\]
\definition{Tautology laws}
\[
P \land \text{(a tautology)} \equiv P
\]
\[
P \lor \text{(a tautology)} \equiv \text{(a tautology)}
\]
\[
\lnot \text{(a tautology)} \equiv \text{(a contradiction)}
\]
\definition{Contradiction laws}
\[
P \land \text{(a contradiction)} \equiv \text{(a contradiction)}
\]
\[
P \lor \text{(a contradiction)} \equiv P
\]
\[
\lnot \text{(a contradiction)} \equiv \text{(a tautology)}
\]
\subsection{Variables and Sets}
\[
\mathbb{R}^{+,-} = \{x | x \text{ is a real number}\}
\]
(a number that can be written on the number line)
\[
\mathbb{Q}^{+,-} = \{x | x \text{ is a rational number}\}
\]
(a number that can be written as a fraction $\cfrac{p}{q}$)
\[
\mathbb{Z}^{+,-} = \{x | x \text{ is an integer}\} = \{ \ldots, -3, -2, -1, 0, 1, 2, 3, \ldots \}
\]
\[
\mathbb{N} = \{x | x \text{ is a natural number}\} = \{0, 1, 2, 3, \ldots \}
\]
\subsection{Operations on Sets}
\definition{Intersection of two sets A and B}
\[
A \cap B = \{ x | x \in A \text{ and } x \in B \}
\]
\definition{Union of two sets A and B}
\[
A \cup B = \{ x | x \in A \text{ or } x \in B \}
\]
\definition{Difference of A and B}
\[
A \setminus B = \{ x | x \in A \text{ and } x \notin B \}
\]
\definition{A is a subset of B if every element of A is also an element of B}
\[
A \subseteq B
\]
\definition{A and B are disjoint if they have no elements in common}
\[
A \land B = \emptyset
\]
\subsection{The Conditional and Biconditional Connectives}
\definition{Conditional laws}
\[
P \to Q \equiv \lnot P \lor Q
\]
\[
P \to Q \equiv \lnot (P \land \lnot Q)
\]
\definition{Contrapositive law}
\[
P \to Q \equiv \lnot Q \to \lnot P
\]
\section{Quantificational Logic}
\subsection{Quantifiers}

\end{document}